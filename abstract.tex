
Collaborative information analysis is a form of sensemaking that involves modeling and representation of complex information space through synchronous and asynchronous team interactions over extended periods. Effectiveness of collaboration determines team performance, which directly impacts the justice of decisions made or solutions proposed. Effective collaboration is challenging because extra efforts are required to develop and maintain team awareness on top of information analysis task, which itself demands a high need of cognition.

A lot of research has been conducted to support collaboration and information analysis \textit{separately}. For example, numerous tools and techniques have been developed to facilitate the processing and visualization of information analysis, but the majority are designed for individual use rather than for collaborative use. On the other hand, collaborative tools, also known as groupware, are developed to improve team effectiveness, yet they often lack support for advanced information modeling and representation. A gap exists between the two research areas, a design space yet to be explored when a \textit{team} is engaged in collaboratively analyzing a set of data. Simply applying design implications from both areas together should not work because we must address the tension of need for cognition in both the task of collaboration and the task of information analysis. To some extent, the task of collaborative information analysis is a different activity from either. People are engaged in a different workflow and face new challenges, which this research tries to understand and support.

Evaluating effective tool support for collaborative information analysis is another multifaceted and complex problem. Researchers often reduce the complexity of the analytic task for the sake of ease of measurement. In practice, it is challenging to model complex analytic scenarios in lab studies which span only a couple of hours, and real cases and professional analysts are often limited to access in reality. However, the level of task complexity directly determines the need for cognition in analysis and influences the collaboration strategy teams will employ. This research addresses the challenge of empirical evaluation and targets at situations in which professionals need to both analyze complex information and collaborate about decisions.

My research starts with a task analysis of an example of collaborative information analysis in the real world: an undergraduate course of intelligence analysis at Pennsylvania State University. I described student analysts' workflow and their team behavior with current tooling. Specifically, the study observes that structured techniques are frequently employed but lack serious collaborative support. Based on the observation, five design objectives are proposed for a better collaborative tool. These objectives drive the design and development of a new tool out from this dissertation, \textit{CAnalytics}, an integrated analytic workspace that supports real-time collaborative data annotation for information modeling and collaborative data visualization for information analysis. 

\textit{CAnalytics} is evaluated in two classroom studies, in which students are being trained to become professional information analysts. I did a quantitative analysis on system logs and student reports, as well as a qualitative analysis of questionnaires. The first study emphasizes assessment of integrating structured information modeling and visualization in a single collaborative workspace. I analyze different team behaviors on multiple dimension, and their interaction with team performance. The second study focuses on supporting a higher-order information analysis activity, i.e. collaborative hypothesis development, using a structured approach. Both studies contribute to the understanding of analysts' team behavior in collaborative information analysis and the role of computing tools in support of both collaboration and information handling.

In summary, this dissertation contributes an understanding of how analysts use computing tools for collaboratively analyzing information in the real world. The research produces a collaborative visualization tool that leverages structured techniques from the intelligence community as well as design knowledge of team awareness from the CSCW (Computer-Supported Collaborative Work) community. The classroom studies evaluate design choices and help identify emerging design challenges. The dissertation ends with design implications and proposes that structured techniques and collaboration be enablers for each other.