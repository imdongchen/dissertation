\section{Problem Statement}

Study One presents a positive impact on team behavior of an integrated analytic workspace and a structured approach of data annotation and visualization. On top of that, in Study Two, we want to explore the impact of a more structured approach to collaborative hypothesis development. 

Analysts generate multiple hypotheses and find evidence to confirm or disconfirm them. A common technique to evaluate hypotheses is Analysis of Competing Hypotheses (ACH) \citep{Heuer1999}. ACH emphasizes a rational, transparent process of hypothesis analysis. The steps include identifying a complete set of hypotheses, identifying all pieces of evidence, assessing the diagnostic value of each evidence against each hypothesis, and drawing conclusions of the likelihood of hypotheses. ACH guarantees an appropriate analytic process and increases the chance to avoid common analytical pitfalls such as confirmation bias.

In practice, however, we found students rarely came up with a complete set of hypotheses in the very beginning. Instead, they began with reading documents and got familiar with the suspects. They would propose a hypothesis in the process of extracting and marshaling evidence. They would share their hypothesis with teammates or document them in a note. With evidence accumulating, the team collaboratively refine the existing hypotheses and propose alternative hypotheses. The process was iterative, with hypotheses constantly being refined and evolved.

ACH demands a high expertise bar for analysts. Users are required to identify a complete set of mutually exclusive hypotheses at the very beginning, which is difficult for people with little expertise and experience in the domain. Further, hypotheses tend to evolve as analysis proceeds. A hypothesis valid in the beginning may no longer be of value later, or two seemingly separate hypotheses in an early stage of analysis could be combined in a way to better explain the situation later on. The assumption ACH makes that all hypotheses and evidence are identified and set in the very beginning limits the dynamic development of analysis.

We believe that analysts should not only share their hypotheses but should also collaboratively develop and refine their hypotheses. ACH tend to treat hypotheses as complete, well-defined objects. Analysts decide a set of alternatives and evaluate the evidence against them. In reality, however, hypotheses tend to change and evolve as analysts collect more information and gain a deeper understanding of the situation. Hypotheses that might be valid at the beginning of a process, may come to be inappropriate, or incorrect as the analysis progresses. While simply sharing hypotheses is beneficial to analysis teams, its positive impact is limited to an extent as it is not completely leveraging the various expertise of a team. Where instead, the combining and evolving of hypotheses from and by multiple members of the team (where the evidence that the hypothesis is based on is included in the hypothesis object) creates an environment where the team is more than just a sum of its parts.





