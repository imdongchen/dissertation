\section{Education in collaborative information analysis}

An important direction of this research is to explore how such a collaborative instrument could be employed to support collaborative learning in education. We studied an introductory intelligence analysis course, but many other courses involve heavy information analysis tasks, in which snippets of information must be annotated and analyzed with respect to concepts, connections, weights of evidence, and pros and cons of opinions. This is similar to generating and evaluating hypotheses in our case. Thus one could imagine courses in business, new media, and those involving heavy literature review adopting this tool similar to the case described here.

The goal is to shape students' behavior towards more collaborative learning. With traditional single-user tools, students often employ a divide-and-conquer strategy; they divide their job responsibility, work individually on their own part, and put the results together in the end for report submission. In our study, we observed that students spontaneously conducted closer collaboration and enjoyed being able to contribute simultaneously.
% cite education paper in intelligence analysis

In addition to shaping the \textit{learning} behavior of students, a collaborative supporting tool can also change how an instructor \textit{teaches} analytic skills. Anecdotal reflections from the instructor in our classroom study emphasized the role of system support in \textit{instructor intervention}. During our classroom study, the instructor frequently went over to the student's desk and checked how they were doing. The instructor commented that he valued students' analytic process as much as their final report. His emphasis on analytic process is consistent with the value of the intelligence community, who claimed that analysts \textit{``should think about how they make judgments and reach conclusions, not just about the judgments and conclusions themselves''} \citep{Heuer1999}. However, the process is hard or costly to capture for assessment in reality. In the context of education, for example, all students are conducting analysis at the same time. Without explicit support, the instructor has limited supervision over the process.

The collaborative learning instrument is likely to provide an opportunity to get a student's analytic traces preserved and assessed because the system has already captured and saved user interactions. These logs are currently saved for research purposes in our study, but potentially they can be displayed to the instructor for assessment purposes. Taking advantage of the system's synchronization capability, student's behavior can be streamed to a ``monitoring dashboard'' in real time, from which the instructor could check student's progress and take early intervention if any team deviates from the expected path.

\section{Summary}

This dissertation contributes an understanding of how analysts use groupware to analyze complex datasets over extended periods. The research presents a design process that identifies design requirements from task analysis, embodies requirements into groupware prototypes, observes team behavior mediated by the tool, and then repeats the process. Through a series of user studies, I investigated how we can better support collaborative information analysis. To this end, I implemented and evaluated two versions of a groupware tool to understand the effects of these design considerations on collaboration. Here is a brief review of the major contributions of the dissertation.

\textbf{C1: A characterization of collaborative information analysis activities.} 
These activities include data modeling, visual analysis, and hypothesis development. Structured techniques such as ACH and link analysis are often employed in the process. Collaboration is often impeded by the limit of concurrency and manual sharing.

\textbf{C2: A fully functional tool that supports collaborative information analysis.}
Features of the tool include integrated support for data modeling, visual analysis, and hypothesis development, a structured approach for data analysis activities, team awareness support, and capture of high-level, semantic usage logs. The architectural design and implementation can help future groupware overcome technical issues in synchronicity enablement and complex state management.

\textbf{C3: A characterization of analysts' team behavior mediated by groupware over extended periods.}
Evaluation of the tool through two classroom studies with analysts-in-training reveals various team behaviors such as filtering and accretion, situational logic and comparison, inferential connection and factual discrete, as well as a loose coupling in data modeling and close collaboration in hypothesis development. The result highlights the importance of providing an integrated workspace that supports the smooth transition of analytic activities.

\textbf{C4: Design guidelines to improve collaborative information analysis support.}
These guidelines include properly representing uncertainty both from data and from teammates, building collapsible views for hierarchical data, treating views as a team resource like data that are sharable, extensible, and reusable, distinguish visible and valuable contributions to encourage high-quality participation, and supporting multiple levels of collaboration closeness throughout the analysis lifecycle. It also calls for a combination of structured technique and awareness technology to better support collaborative information analysis.

