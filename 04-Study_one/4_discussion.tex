\section{Discussion}\label{discussion}

The goal of the study is to understand and support an interleaved, collaborative workflow in information analysis by evaluating tool usage in a natural
environment over multiple usage sessions. Our work responds to calls for an
integrated environment in the intelligence community as well as other data
intensive domains \citep{Shah2014i, Chen2016, Vision2015, Amershi2015, Ware2012}.
The system builds upon prior empirical studies (e.g. \cite{Carroll2013,
	Borge2012,Kang2011,Chin2009}) and embodies their design implications in our
tool. The study also complements research that only tests tools in short term
lab studies (e.g. \cite{Convertino2011,Goyal2016,Hajizadeh2013}) by
investigating tool usage over multiple usage sessions.

This study adopts an \emph{analyst-centered design} approach. A critical
requirement of developing tools that meet user needs is to understand their
needs and practices. When these needs and practices are specialized (as is the
case in intelligence analysis), it is particularly important to include the
target user population in the design process \citep{Scholtz2014}. Our classroom
study provided an opportunity with deep access to analysts in training. These
analysts have been trained with knowledge and skills of intelligence analysis,
and have experience with state-of-the-art analytic tools such as Analyst's
Notebook and PARC ACH. In their reflections, participants often compared
CAnalytics to those tools. Their multi-session usage of CAnalytics also allows
them to adapt to the tool and learn to appropriate it to the best of their team
purpose \citep{Stahl2006}. Therefore, while their feedback is admittedly not the
same as an experienced professional, their feedback does provide a deeper
insight into strengths and weaknesses of the tool than average users.

The study provides encouraging results. Participants appreciated an integrated
environment where they could share raw documents, evidence snippets, views of
evidence and hypotheses in one place. They liked the fact that they could
contribute simultaneously without blocking or interfering each other. Another
collaborative tool is to keep teammates aware of each other's
activities. Staying aware of teammates not only helps establish a common ground
for planning team strategy, but also ensure everyone is following the plan as
expected. Moreover, results suggest the awareness features provide positive
\emph{social facilitation} \citep{Zajonc1965}: individuals found the task
motivating and engaging with awareness of each other's activity. We also
measured collaboration characteristics that impacted team performance, and found
that balanced contribution, peer editing, and earlier switching from modeling to
analysis predicted higher team performance. Most importantly, we documented the
interleaved workflow enabled by the integrated environment, and explored
momentum in modeling and analysis behaviors that drove the activity switching.
Below we discuss design implications that could potentially enable a better
collaborative experience in information analysis tasks.

\subsection{Scaffold a structured interleaved workflow}

An typical misconception about information analysis is that data modeling and
data analysis are two staged activities. This is akin to the \emph{waterfall}
software development model, which features a sequential process that
flows downwards through the phases of requirement conception, software
design and implementation, and testing and maintenance. Critics have
argued against this approach and instead called for an iterative design
process that leads to reframe user requirements,
redesign, redevelopment, and retesting.

Our result demonstrates a similar iterative and dynamic process in information
analysis. The result is striking especially because our participants have been
trained with tools that impose a waterfall model. They could have followed their
old waterfall practice with our tool, yet instead all teams spontaneously
switched to an iterative manner. We projected that modeling on multiple granularities drives analysis on different levels, and uncertainty in analysis pushes analysts back to collecting additional data.

Realizing that, we probably could shape analysis into a more interleaved
workflow with a more structured approach. Annotation is a process in which analysts are constructing their own version of information space on the basis of information provided by the document. The resulted representation of annotation will strongly influence their analysis. Meanwhile, analysis generates hypotheses which will frame how analysts perceive their data. The gap between hypotheses and facts then drives analysts to annotate a new version of information space. Structured techniques such as IEW and ACH help users
perform analysis in a systematic and transparent manner in each well defined activity, yet fall short in guiding analysts in switching between the two activities
\citep{Kang2011}. Our system implements a structured modeling through annotation and
a structured analysis by visualizing data in multiple views. By sharing the same
data structure and consistent user interface, we enable a smooth switching
between the two stages. Our result suggests the role of multilevel modeling and
analysis uncertainty in driving the switch. Based on that, there exist design
opportunities to enable a more interleaved flow. For example, we can build a more structured scaffolding process to bridge modeling and analysis. When a user adds a
new data object, the system could suggest possible connections to existing
evidence in the context of an appropriate level of views, which is likely to
help analysts discover new patterns. When a user creates a new hypothesis with
uncertainty, the system could highlight associated evidence, which would prompt
the analyst to re-examine the data and look for more data. Such scaffolding
provides a basic structure to link stages of analytic activities that analysts
can take on without imposing a specific fixed workflow.

A smoothier interleaved workflow could also be made by increasing team
awareness of partner's modeling and analysis. That is, uncertainty in one's
analysis not only motivates oneself to data modeling, but also drives partners
(who is aware of the what and why of the uncertainty) to collect more data.
Similarly, one's modeling could influence and drive partner's analysis, given
the partner is fully aware of what is modeled and how the new data connects to
the level of existing data. Such ``team-level'' interleaving could make teamwork
more interactive and close coupling, but also requires support of higher
awareness, especially of hypothesis uncertainty and data model context. Improved
design for multi-granularity modeling, uncertainty representation, as well as
team awareness of these features opens up possibilities for coordinating
interleaving at the team level. We discuss these design implications in further
detail in the following paragraphs.

We noticed that iterative data annotating activities indicate analysts must be in control of the data modeling process.
Depending on the questions and hypotheses in analyst's mind, an analyst may want to model different types of evidence. Therefore the analyst should always be in control of evidence modeling, while the computational method only \emph{facilitates} data modeling, e.g. ease data input, hide complexity of format structuring. 



\subsection{Represent uncertainty in collaborative analysis }

Our finding suggests that uncertainty in collaborative analysis requires deeper understanding and brings more design challenges and opportunities than in traditional data visualization. Uncertainty is prevalent in data. For example, the credibility and reliability of evidence vary depending on its source (witness reports vs. police reports) and collection means (video vs. narrative). Most research efforts focus on uncertainties explicit in data and how to represent them \citep{Pang1997,Zuk2008}, but there are other sources of uncertainty as well. For example, multiple interpretations might be equally plausible towards the same piece of information. Such uncertainty, also known as ambiguity \citep{Nowak2020}, arises when individuals create annotations based on their own knowledge and experience. In case of collaborative data analysis, teammates become a source of uncertainty when they share annotations and hypotheses with uncertainty. 
It is often easier (and people are more inclined) to represent information of \textit{certain}, but that of uncertain tends to be lost or vague due to the lost of the context \citep{prue2014overcoming}. This proposes new research questions: how to allow and encourage analysts to represent uncertainty? How to represent collaborator-generated uncertainty together with data-bearing uncertainty? Shall we distinguish them or treat them equally? 

In addition to representing uncertainty, reasoning with uncertainty poses new challenges \citep{MacEachren2015}. Nuance exists in understanding and reasoning with uncertainty among teammates. Collaborators make their own evaluation of uncertainty, based upon their unique knowledge, experience, role, as well as trust in the source teammate \citep{Chin2009}. It is critical that teammates are aware of the existence of uncertainty when they share their sensemaking, and reach a team-level agreement of the information gap. \cite{Graves2000} characterized intelligence analysis as a progression where the analyst's certainty grows over time. And we can probably describe collaborative information analysis as a progression where a team of analysts establish awareness of certainty, and achieve a growing team agreement of certainty. 

While uncertainty poses a big challenge to analysts, we  perceive it as a design opportunity at the same time to identify emerging needs for team collaboration. When uncertainty arises, team attention should be obtained and teammates should communicate and establish an agreement on the representation of the uncertainty. Tools should be designed to make uncertainty easy to be represented, shared and discovered, and even proactively identify and notify uncertainty in team communication. 

In our study observation, teams spontaneously employed two different approaches: 
1) represent uncertainty on top of data and 2) separate uncertainty from data. In fact the research community does not have an agreement whether it is better if uncertainty is visualized or suppressed, or under what conditions it is better \citep{Maceachren2005}. We should probably provide users with control in deciding how to display uncertainty. For example, users can \emph{filter} by uncertainty so that they can
choose to consider only evidence of certainty, or to review all
inferences and decide what extra data is needed to consolidate them. A a richer graphic language is needed to encode uncertainty in the analytic view \citep{Bertin1983} and to make reference to uncertainty in team discussion. 

\subsection{Build collapsible views for multi-level modeling}

We observed that analysts built data models in multiple granularities and coordinated among multiple levels of details. For example, analysts jumped from digging into details of a single event, to comparing between two
events, and to overviewing all robberies as a complete story. When sharing these data, a critical requirement is to represent them in an appropriate context in order to ensure teammates understand them correctly. A collapsible data view could be a solution to accommodate such
multi-level modeling. This can help draw teammate's attention to a specific item
while keeping the global context available. Analysts can focus on a certain
level of detail at a time while conveniently switching between levels. A
collapsible view also reduces the problem of cluttered view when data volume
increases.  and when analysts dig into greater
details (e.g.~representing suspect's all actions to identify patterns of common
actions in two robberies). An analyst can overview all robberies and only unfold
detailed actions when investigating into a specific robbery.

\subsection{Share views as team resource}

View sharing provides an anchor for team communication and is important for sharing and understanding analytic result
\citep{Morton2014a} and. A common approach to view sharing is to enable a public view which always
keeps synchronized for all teammates \citep{Convertino2011,Greenberg1990}.
However, a single public view does not meet the need in dynamic information analysis.
Analysts often share multiple views in an iterative workflow. Most views are for evidence keeping, and some views are more important that may be repeatedly used in a report. Other views may be reused by teammates and further developed.

We propose that views should be treated as a \emph{team resource}. Just
like data, views can be sharable, extensible, and reusable. For example, an analyst
can save their current view as a shared resource when they feel it useful to
collaborators. Other people can reuse the view to their need. To facilitate the reuse of views, \cite{Nobarany2012} designed a recommendation system to help discover possible useful views from teammates. Shared views
should be interactive rather than static images, so that analysts can perform
all interactions including filtering and highlighting, and are able to evolve
the view with collective team efforts, a critical requirement emphasized in
\cite{Carroll2013}.

A public workspace to
which partners can refer sets up a common ground for discussion. Synchronous
groupware used to impose “what you see is what I see” (WYSIWIS) (e.g. \cite{stefik1987wysiwis}),  in which all team members share the same view. However, this design
blocks production as only one member is able to control the workspace in one
turn. Relaxed-WYSIWIS is thus proposed in contrast to the previous
strict-WYSIWIS, allowing people to navigate the workspace independently \citep{Gutwin1998h}. 
Indeed, in observation of empirical teamwork, participants
were found to keep private notes while contributing to shared note-taking \citep{Borge2012}.

\subsection{Distinguish visible vs. valuable contributions}

Finally, we noted cases where teams created far more entities than needed with an
accretion strategy. Strikingly, while similar data modeling strategy was
reported in the paper prototype study \citep{Carroll2013}, users with CAnalytics
seemed far more tempted to accretively add information, with far more entities
and cluttered views. For example, the team with most entities created triple the number of entities as 
other teams did in our study, much more than the difference
in the paper prototype study. Why did this happen? We guess both the context of
classroom study and the system design contributed. Unlike in the lab study where
teams are temporarily assembled, teams in a class evaluate peers either
consciously or unconsciously and value how themselves are being evaluated. Such
social pressure motivates individuals to make contributions, and indeed to make
\emph{visible} contributions, more than \emph{valuable} contributions. That is,
participants noticed that their work activity was visible to their partners, and
accordingly prioritized doing more visible work over doing less visible work. In
some cases, this led to a new problem of easy and less valuable contributions
that were highly visible - such as creating and therefore sharing data entities
that were not particularly important, and subsequently made data models seem
cluttered. For example, creating and therefore sharing an entity gets
immediately notified to the team whereas weighing the importance and relevance
of an entity goes silent in the system. We need to investigate approaches to
making significant contributions more visible, or perhaps making it more
immediately visible that less important contributions are indeed less important.