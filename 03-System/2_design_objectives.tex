\section{Design Objectives}\label{design-objectives}

Based on our task analysis, we come up with five design objectives for our collaborative analytic tool.

\subsection{Objective 1. Provide an integrated, dynamic  analytic workspace}

While functions can be provided to support a single analytic activity (i.e. data annotation alone, or data visualization alone, and indeed a lot of tools exist to support either activity separately), a critical design objective is to integrate these functions in a single workspace. This is more than simply put together these functions; rather, a smooth transition should be enabled between activities. The goal should be achieved in at least two aspects: (1)
the same data pool and structure is utilized in all analytic activities. Data
created in data annotation can be visualized in analysis, and utilized in
hypothesis development; (2) The system provides a consistent user experience in
different activities. Analysts should be confronted with a consistent user
interface and interaction throughout the process.


\subsection{Objective 2. Support structured data annotating and visualization techniques}

Our design should allow for a structured approach to modeling and analyzing data. Analysts spend a large amount of time and effort annotating critical information from text documents. The essence of an annotation is to turn a set of unstructured, noisy data into structured, useful information while preserving its source. We want to provide a computational method that facilitates the annotation process so that analysts can focus on information itself rather than the logistics. 

The annotated data objects should then allow for a structured analysis of patterns from multiple aspects through visualization. For example, temporal visualization helps analysts keep track of the temporal evolution
of events. Spatial visualization helps identify patterns in the same area. Interactions should be provided so that analysts can coordinate and arrange these views into a bigger picture. These closely presented views form into a proximity-based projection \citep{Kang2014a}, where each view presents a related insight. Analysts can easily navigate these evidence views and marshal them together to support hypothesis development \citep{Pirolli2005}. 

\subsection{Objective 3. Support information sharing and team awareness}

Collaboration is a process of management of team resources and group process. And a pre-requirement for group management is to stay aware of them. We propose to enable real-time data sharing and explicit awareness support.
Analysts should be able to share source documents, annotated data objects, patterned
views, and hypotheses and insights in real-time. Partners can verify who
identified and shared the information and can validate against the source
document to confirm sanity. One’s activities should be transparent to teammates
without extra effort so that everyone is aware of other’s activities.

\subsection{Objective 4. Enabling collaborative hypotheses development}

A more advanced objective is to enable hypothesis development and particularly collaborative development on a team basis. While simply
sharing each teammate's hypotheses is beneficial, its positive impact is
limited to an extent as it is not completely leveraging the various expertise
of a team. Where instead, the combining and evolving of hypotheses from and by
multiple members of the team (where the evidence that the hypothesis is based
on is included in the hypothesis object) creates an environment where the team
is more than just a sum of its parts. We will help analysts to share their
insights/hypotheses with supporting visualization early in the process. To
impart knowledge, visualizations conveniently improve the understanding of
complex problems. When users are exploring the visualization and find
insight, they will be able to share the insight together with the grounding
visualization. The visualization will be interactive rather than a static
screenshot, keeping the ad hoc visualization state. Insofar as stating unknowns
is a difficult task, at least making what analysts do know more explicit and
the assumptions that they made increases a large amount of transparency into
their analysis

\subsection{Objective 5. Capture team behavior}

Other than assisting analysts in accomplishing the task more efficiently, we develop the tool as an instrument to gain a deeper understanding of team behavior. The computing tool can capture and log all interactions analysts have made. These interaction data provide a promising window into which researchers can identify insights in analysis patterns on both an individual level and a team level. Compared to video recording, computing logs have user interaction data already in structure and ready for data analysis without the hassle of human annotating. Log analysis is also easier to scale up to analysis of longer-term interactions than video analysis (e.g. in Study One, we have 73 persons (25 teams),  each person working for about 6 hours, adding up to 428 hours!)
