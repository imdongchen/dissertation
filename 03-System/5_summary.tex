\section{Summary}

To understand how teams will work with a collaborative analytic tool, a fully functional system is needed. To meaningfully inform the design process, I started with a task analysis of collaborative information analysis. I worked with a lab that specializes in Intelligence Analysis and conducted an observational study of an undergraduate course that teaches Intelligence Analysis. I observed three repeated activities: data wrangling/modeling, data analysis, and hypothesis development. To answer an analytic question, analysts transform and map data from raw documents to another format, and feed it to a corresponding downstream analysis, based on which hypotheses are formed and reported. While collaboration is essential, it is often blocked by the limit of tools: teams either have no synchronous writing access to the same file or simply put together individual findings in the end. 

Based on my observation, I defined five design goals and developed CAnalytics, a Web application that integrates structured data annotation, visual analysis, and hypothesis development in one place while supporting information sharing in real-time. The chapter walks through its features, explains major technical challenges, and describes the implementation details. 

In the next two chapters, I will present two classroom studies I conducted in two Intelligence Analysis courses at Pennsylvania State University. The purpose of these studies is to demonstrate how well the integration of structured techniques and awareness support could fit within the real application domain and how teams would behave in a natural environment.